\documentclass{article}
\usepackage[utf8]{inputenc}
\usepackage[english]{babel}
\usepackage{amsthm, amssymb, amsmath, amsfonts, extramarks}
\usepackage{gen}


\title{Definitions}
\author{Bailey Wickham \\ MATH 412}

\date\today

\newenvironment{homeworkProblem}[1]{
    \section*{Problem #1}
}


\begin{document}
\maketitle

\section{Chapter 1}
\subsection{Maximum}
Let $A \subset \bR, m \in \bR$. We say that the \textbf{maximum} of $A$ is $m$ if
\begin{enumerate}
    \item for all $x \in A, m \ge x$
    \item $m \in A$
\end{enumerate}

\subsection{Upper bound}
Let $A$ be a nonempty subset of \bR, let $u \in \bR$. $u$ is an \textbf{upper bound} if for all $x\in A$ $x \le u$

\subsection{Supremum}
Let $A$ be a nenempty subset of \bR, Let $l \in \bR$. $l$ is the \textbf{supremum} of $A$ if for all $a \in A, a \le l$ and for all $m \in \bR$ if all $a \in A, a \le m$ then $l\le m$. $l$ is the least upper bound.

\subsection{Minimum}
Let $A \subset \bR, m \in \bR$. We say that the \textbf{minimum} of $A$ is $m$ if
\begin{enumerate}
    \item for all $x \in A, m \le x$
    \item $m \in A$
\end{enumerate}

\subsection{Infimum}
Let $A$ be a nenempty subset of \bR, Let $l \in \bR$. $l$ is the \textbf{infimum} of $A$ if for all $a \in A, a \ge l$ and for all $m \in \bR$ if all $a \in A, a \ge m$ then $l\ge m$. $l$ is the least upper bound.

\subsection{Bounded Above}
A subset $A \subset \bR$ is \textbf{bounded above} if there exists a supremum for $A$.
\subsection{Bounded Below}
A subset $A \subset \bR$ is \textbf{bounded below} if there exists a infimum for $A$.

\subsection{Bounded}
A set is \textbf{bounded} if it is bounded below and bounded above

\subsection{Bijection}
A function $f$ is a bijection if it is one to one and onto.

\subsection{Cardinality}
Sets $A,B$ have the same cardinality if there exists a bijection between them.

\subsection{Finite}
Let $A$ be a set, $A$ is finite if there exists an $n \in \bN$ s.t.$A$ has card $n$

\subsection{Countable}
A set is countable if there is a bijection between $\bN$ and $A$.

\subsection{Sequence}
A sequences is a function whose domain is \bN

\subsection{Convergence}
A sequence $(a_n)$ converges to a real number a if for every positive number $\epsilon$ there exists an $N \in \bN$ such that for all $n \ge N$, then $|a_n - a| < \epsilon$

\subsection{$\epsilon$-Neighborhood}
$V_\epsilon (a) = \{x \in \bR : |x-a| < \epsilon\}$

\subsection{Topological Convergence}
A sequence $(a_n)$ converges to $a$ if there exists an $n \in \bN$ such that all terms after $n$ are in an $\epsilon neighborhood$

\subsection{Bounded}
A sequence is bounded iff the set of all elements of the sequence is bounded.

\subsection{Eventually}
A sequence $(a_n)$ is eventually in a set $A$ if there exists an $N$ s.t. $a_n \in A$ for all $n \ge N$

\subsection{Frequently}
A sequence $(a_n)$ is eventually in a set $A$ if for every $N$  there exists an $n \ge N$ s.t. $a_n \in A$.

\section*{Chapter 2}
\subsection*{2.4 The Monotone Convergence Theorem.}

\subsubsection*{Increasing}
A sequence is increasing if for all $n \in \bN$, $a_n \le a_{n+1}$
\subsubsection*{Decreasing}
A sequence is decreasing if for all $n \in \bN$, $a_n \ge a_{n+1}$

\subsubsection*{Monotone}
A sequence is monotone if it is increasing or decreasing.

\subsubsection*{Infinite Series}
An infinite series is a expression in the form of:
\begin{equation*}
\sum_{n=1}^{\infty} b_n = b_1 + b_2 + ...
\end{equation*}
The sequence of partial sums $(s_m)$ by $s_m = b_1 + b_2 + ... + b_m$. We say the series converges to $B$ if the sequence of partial sums converges to $B$, denoted:
\begin{equation*}
\sum_{n=1}^{\infty} b_n = B
\end{equation*}

\subsubsection*{Geometric Series}
Let $r,a \in \bR$. A geometric series is of the form
\begin{equation*}
\sum_{n=1}^{\infty} ar^n
\end{equation*}

\subsection*{2.5 Subsequences}
\subsubsection*{Subsequence}
Let $(a_n)$ be a sequence of real numbers and let $n_1 < n_2 ...$ be an increasing sequence of natural numbers.
Then the sequence $(a_{n_1}, a_{n_2},...)$ is a subsequence of $(a_n)$ and is denoted by $(a_{n_k})$.

\subsection*{2.6 The Cauchy Criterion}
\subsubsection*{Cauchy Sequence}
A sequence $(a_n)$ is called a Cauchy sequence if for every $\epsilon > 0$ there exists an $N \in \bN$ such that whenever $m,n>N$ it follows that $|a_n -a_m|<\epsilon$

\subsection*{2.7 Series}
\subsubsection*{Absolute and Conditional Convergence}
if the series $\sum_{n=1}^{\infty} |a_n|$ converges we say the original series $\sum_{n=1}^{\infty} a_n$ converges absolutely. If the original series converges but the series of absolute values diverges, then we say the sequence converges conditionally.

\subsubsection*{Rearagements}
Let $\sum_{n=1}^{\infty} a_n$ be a series. A series $\sum_{n=1}^{\infty} b_n$ is called a rearrangement of the original series if there exists a bijection between the the series.

\section{Intro to Topology}
\subsection*{3.2 Open and Closed Sets}
\subsubsection*{Open}
A set $O \subseteq \bR$ is open if for all points $a \in O$ there exists an $\epsilon$-neighborhood $V_{\epsilon}(a) \subseteq O$.

\subsubsection*{Limit Point}
A point $x$ is a limit point of of a set $A$ if every $\epsilon$-neighborhood $V_\epsilon (x)$ of $x$ intersects $A$ at some point other than $x$.
Alternatively, $x$ is a limit point if and only if $x=lim(a_n)$ for some sequence $(a_n)$ contained in $A$ satisfying $a_n \ne x$ for all $n \in \bN$.

\subsubsection*{Isolated Point}
A point $a \in A$ is an isolated point if it is not a limit point.

\subsubsection*{Closed Set}
A set $F \subseteq \bR$ is closed if it contains its limit points. Alternatively, a set is closed if and only if every Cauchy sequence contained in $F$ has a limit that is also in $F$.

\subsubsection*{Closure}
The closure of a set $A \subseteq \bR$, denoted $\overline{A}$ is the set $A \cup L$ where $L$ is the set of all limit points of $A$

\subsection*{3.3 Compact Sets}
\subsubsection*{Compact}
A set $K \subseteq \bR$ is compact if every sequence in $K$ has a subsequence converging to a limit also in $K$. A set is Compact iff it is closed and bounded.

\subsubsection*{Bounded}
A set $A \subseteq \bR$ is bounded if there exists $M > 0$ such that $|a| \le M$ for all $a \in A$.

\subsubsection*{Open Cover}
An open cover of A is a collection of open subsets of $A$ such that their union is all of $A$. A finite subcover is a finite subcollection of open subsets of $A$ that still contains $A$ in the union.

\subsection*{3.4 Connected Sets}
\subsubsection*{Seperated and Disconnected Sets}
Two nonempty sets $A, B \subseteq \bR$ are seperated if $\overline{A}\cap B = \overline{B} \cap A = \{\}$. A set is disconnected if it can be written as the union of two seperated sets.

\section{Functional limits and continuity}
\subsection*{4.2 Functional Limits}
\subsubsection*{Functional Limit}
Let $f:A\to \bR$, and let $c$ be a limit point in the domain of $A$ we say that $lim_{x \to c}f(x)=L$ provided that for all $\epsilon > 0$ there exists a $\delta > 0$ such that whenever $0 < |x-c| < \delta$ it follows that $|f(x) - L| < \epsilon$. Alternatively, if for every epsilon neighborhood $V_{\epsilon}(L)$ there exists a $V_{\delta}(c)$ such that for all $x \in V_{\delta}c$ different than c it follows that $f(x) \in V_{\epsilon}(L)$

\subsection*{4.3 Continous Functions}
\subsubsection*{Continuity}
A function $f:A\to \bR$ is continuous at a point $c \in A$ if for all $\epsilon > 0$ there exists a $\delta > 0$ such that whenever $0 < |x-c| < \delta$ it follows that $|f(x) - f(c)| < \epsilon$.

\subsection*{4.4 Continous Functions on Compact Sets}
\subsubsection*{Uniform Continuity}
Let $f:A\to \bR$, $f$ is uniformly continuous on $A$ if for all $\epsilon > 0$ there exists a $\delta > 0$ such that whenever $x,y \in A$, $|x-y| < \delta$ implies $|f(x) - f(y)| < \epsilon$.

\subsection*{4.5 The IVT}
\subsubsection*{IVP}
A function has the intermediate value property on $[a,b]$ if for all $x<y$ in $[a,b]$, and all $L$ between $f(x) and f(y)$it is always possible to find a point $c \in (x,y)$ where $f(c)=L$

\section{Derivatives}
\subsection*{5.2 Differentiability}
\subsubsection*{Differentiability}
Let $g:A\to \bR$ be a function defined on an interval $A$. given $c \in A $, the derivative of $g$ at c s defined by
$g'(c) = lim_{x-c} \frac{g(x) - g(c)}{x-c}$
we say that $g$ is differentiable at $c$, or on $A$ if $g$ is differentiable over all of $A$.

\subsection*{infinity}
given $g:A \to \bR$ and a limit point c of A, we say that $lim_{x\to c} g(x) = \infty$ if for every $M>0$ there exists a $\delta >0$ such that whenever $0 < |x-c| < \delta$ it follows that $g(x) > M$.



\end{document}

