\documentclass{article}
\usepackage[utf8]{inputenc}
\usepackage{amsthm, amssymb, amsmath, amsfonts, extramarks, gensymb, enumitem, tikz-cd}
\usepackage{general-poly}

% For creating checkmarks
%
% For a task do:
%   \item[\done]
%   \item[\wontfix]
% to insert checkmark or cross.
\newlist{todolist}{itemize}{2}
\setlist[todolist]{label=$\square$}
\usepackage{pifont}
\newcommand{\cmark}{\color{red} \ding{51}}%
\newcommand{\xmark}{\ding{55}}%
\newcommand{\done}{\rlap{$\square$}{\raisebox{2pt}{\large\hspace{1pt}\cmark}}%
\hspace{-2.5pt}}
\newcommand{\wontfix}{\rlap{$\square$}{\large\hspace{1pt}\xmark}}

\title{Rediscovery of $\mathbb{H}^\times$}
\author{Frost}
\date{June 2020}

\begin{document}

\maketitle
\section{Table of Contents}
This is all over \bR (for now)
\begin{itemize}
    \item [x] Intro to Quaternions
    \item Moduli space \bC->\bH
    \item Conjugacy Classes for \bH and for embeddings of \bC
    \item Generate picture? (2 sphere)
    \item Wedderburn's Theorem
    \item Functor $k->A_{a,b}$
    \item $A_{a,b}$, give cases of  a,b over R
    \item Describe C, RxR, nilpotent case
    \item Generate picture of moduli space
    
    \item Short Exact Sequence (This might be able to generalize to k). [what are short exact sequences?]
    \item Conjugacy of types of planes {\color{blue} We still have not written this up formally}. 
    \item Sphere moduli space of the planes (and probability distribution of plane types within isomorphism classes.ntro
\end{itemize}

Short Exact Sequence (This might be able to generalize to k). 
Conjugacy of types of planes {\color{blue} We still have not written this up formally}. 
Sphere moduli space of the planes (and probability distribution of plane types within isomorphism classes.

Are we doing anything over arbitrary fields $k$? \bQ? {\bQ}_p?

\section{Paper/Poster Todo}
\begin{todolist}
    \item[\wontfix] Check the sphere of planes for different choices of a, b. More generally, check that it is preserved within the isomorphism class.
    \item Create mathemetica plots of the two sphere's of planes.
    \item Create mathematica plots of the different plane types (identifying the orthonormal unit to the identity as well as the sqrt(-1)'s, idempotent's, and nilpotent's)
    \item Determine relevant graphics/pretty pictures to make our poster look real good
\end{todolist}

\section{Chaotic TODO}
\begin{todolist}
    \item Find cosets of $S^3$/$S^1$ 
    \item Write up Orbits and Stabilizers 
    \item Describe Orbit of element in H, stabilizer of element. 
    \item[\done] Zero divisors correspond to non division rings (Yes for generalized quaternions)? 
    \item Write up Moduli space for $\mathbb{R} \longrightarrow \bH$, $M_2(\mathbb{R})$, $\bR \times \bR$, nilpotent case. 
    \item Prove (R[x]/p(x)) quotient rings entirely determine sub algebras, prove $\bR \times \bR$, nilpotent, \bC are all and only cases. Prove multiplication works.
    \item Write up proof for no commutative sub alg of dim 3, proof for dim 4.  hurwitz thm?
    \item Understand Functor of Points [lol]
    \item[\done] Decide if Functor of points should be included in the writeup. (yes)
    \item Write up proof of existence of proof of wedderburn?
    \item Show cases of generalized quaternions, what choice of  $a, b, k$ does what. What about over $Q_p$
    \item Possibly include R and Qp solutions to norm poly => M2R or H (Probably not)
\end{todolist}

\section{Questions: 7/24 Meeting}
\begin{itemize}
    \item Can no longer reduce to $SL_2$ without now having two orbits ($\pm$)
    \item How does the moduli space go from a sphere in the Quaternions to a separated plane in       $M_2(\bR)$
    \item Are there $a, b$ such that the variety of the norm polynomial has nontrivial solutions over \bR but not over \bQ? Try to find. Guess coprime numbers - Ian 
\end{itemize}

\section{Notes: 7/27 Meeting}
\begin{todolist}
    \item Have two orbits of $M_2(\bR)$ if conjugating group is $SL_2(\bR)$. Believe that any element of -1*SL2(k) sends the two orbits to each other. Need to check that this is the case.
    \item[\done] In the Split Plane and the Degenerate Plane we want to find the orthogonal unit to the 1 of the embedded real line. (Got around this by using the pure imaginary elements.
    \item[\done] Can generate a moduli space of all of the planes inside of $M_2(\bR)$ which is an $S_2$ worth. Orthogonal complement of the real line inside of $M_2(\bR)$ is a 3 dimensional space, and the collection of points of distance one from this origin in the space must each define a plane.
\end{todolist}

\end{document}
