\documentclass{article}
\usepackage[utf8]{inputenc}
\usepackage[english]{babel}
\usepackage{amsthm, amssymb, amsmath, amsfonts, extramarks}


\title{PL}
% Probably bad format to put this here
\author{Bailey Wickham \\ CSC430}

\date\today

\newenvironment{homeworkProblem}[1]{
    \section*{Problem #1}
}


\begin{document}
\maketitle

\section{Criteria}
Here at Wickham enterprises, we are interested in creating the next mobile operating system. Why we think we will be successful at this where others have failed, and justifying the choice to roll our own OS are good questions, but they are out of the scope of this memo.

To decide what is important in a new operating system, we first look at other current operating systems and look at their goals. For traditional operating systems we have Linux/Unix/BSD... Windows, and MacOS. One major goal of each of these platforms is stability. Stability is needed for developers to be able to write applications for the end user on top of your OS, to make your OS useful. Another goal of these systems is efficency. We want our programs to run fast, the faster the better. Another major goal is security, each of these operating systems should be secure from attackers, a goal which is rarely met. The programming language(s) used should assist with these goals. This means our language should be stable, we shouldn't be using the newest language maintained by a single person at a startup. The language should be fast, as fast as possible without sacrificing developer time. Ideally, the language should also be secure, but as we see with the current set of OSs this isn't always the case. Additionally, we are developing a mobile operating system which probably is using the ARM ISA, so the compilier or interpreter should have arm support.

Now that we have a vauge set of requirements, we can propose a language. Looking at the other OSs, all are implimented in C or a C variant. Linux is C, Windows has a lot of C++ but the kernel is C, MacOS is C, objective-c and swift (from my memory). Realisticly, the only languages I would use for a new operating system are C or C++. These languages have the history, the efficency, and the power to write an OS. Both of these languages are very fast and do not have garbage collection. A new language I would consider using is Rust. Rust has no garbage collection which improves the performance. It also has better guarantees for memory saftey, one of the largest problems with C and C++. One thing rust lacks is the history to compete with C and C++. In class we use C89, Rust wasn't created until 2010. This is a problem because there are still major changes to the language being made. Further, there are feer developers with experinece needed to write an OS.

One of my favorite languages which I wouldn't use is golang. While golang is a (somewhat) low level staticly typed language, it still has a garbage collector and produces fairly large binaries. Golang excels in parellel server programming which is not the goal of a mobile OS. Another language I like but wouldn't use is python. Dynamic typing in the kernel would be a nightmare and anyone who is developing their own OS probably has the resources to use a less developer friendly language like C.

An example of something similar is Fuchsia OS by Google which uses their custom Zircon kernel but it is still in development.

\end{document}

